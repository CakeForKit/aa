\documentclass[a4paper, 14pt]{extreport}

% \usepackage{tocloft} % для настройки внешнего вида оглавления, списка рисунков и списка таблиц

\usepackage{cmap} % Улучшенный поиск русских слов в полученном pdf-файле
\usepackage{tempora} % для работы со шрифтами. Он предоставляет доступ к шрифту Tempora, который является альтернативой шрифту Times New Roman.
\usepackage[T2A]{fontenc} % Поддержка русских букв
\usepackage[utf8]{inputenc} % Кодировка utf8
\usepackage[english,russian]{babel} % Языки: английский, русский
\usepackage{pdfpages} % добавить готовые PDF-страницы
\usepackage{placeins} % \FloatBarrier % Позволяет устанавливать барьер для плавающих объектов. После вызова этой команды LaTeX не будет размещать плавающие объекты дальше этой точки в документе, что помогает удерживать таблицы и фигуры в конкретном разделе или подразделе.
 \usepackage[figure,table]{totalcount} % для подсчета общего количества определенных элементов в документе (в вашем случае - рисунков и таблиц).
 \usepackage{lastpage} % для получения номера последней страницы документа. 

\usepackage{amssymb,amsfonts,amsmath,mathtext,cite,float} % доп вещи для математики и списков
\usepackage{letltxmacro} % безопасный способ переопределения LaTeX-команд
\usepackage{flafter} % для контроля размещения плавающих объектов - гарантировать, что плавающие объекты появятся только после их упоминания в тексте


\usepackage{geometry}
\geometry{left=30mm}
\geometry{right=10mm}
\geometry{top=20mm}
\geometry{bottom=20mm}

\setlength{\parindent}{1.25cm}

\usepackage{microtype}
\sloppy	% изменяет поведение системы выравнивания текста, делая ее менее строгой в отношении ограничений на переносы и пробелы. 

\usepackage{setspace}
\onehalfspacing % Полуторный интервал

\frenchspacing	% использовать **одинаковые** интервалы после предложений и между словами. 
\usepackage{indentfirst} % Красная строка

% Настройка заголовков
\makeatletter
\renewcommand\LARGE{\@setfontsize\LARGE{22pt}{20}}
\renewcommand\Large{\@setfontsize\Large{20pt}{20}}
\renewcommand\large{\@setfontsize\large{16pt}{20}}
\makeatother

\newcommand{\ssr}[1]{\begin{center}
		\large\bfseries{#1}
	\end{center} \addcontentsline{toc}{chapter}{#1}  }

\usepackage{titlesec}
%\titleformat{команда}{формат}{метка}{отступ}{до-кода}[после-кода]
%- **команда**: Уровень заголовка, который вы хотите настроить (например, `\chapter`, `\section`, `\subsection` и т.д.).
%- **формат**: Форматирование самого заголовка (например, размер шрифта, стиль начертания).
%- **метка**: Текст или номер, отображаемый перед заголовком.
%- **отступ**: Горизонтальное расстояние между меткой и заголовком.
%- **до-кода**: Команды, выполняемые непосредственно перед текстом заголовка.
%- **после-кода** (необязательный): Команды, выполняемые непосредственно после текста заголовка.
\titleformat{\chapter}{\large\bfseries}{\thechapter}{14pt}{\large\bfseries}
\titleformat{name=\chapter,numberless}{}{}{0pt}{\large\bfseries\centering}	% к команде `\chapter*{...}`.
\titleformat{\section}{\large\bfseries}{\thesection}{14pt}{\large\bfseries}
\titleformat{\subsection}{\large\bfseries}{\thesubsection}{14pt}{\large\bfseries}
%\titlespacing{команда}{отступ слева}{отступ перед заголовком}{отступ после заголовка}[отступ справа]
\titlespacing{\chapter}{12.5mm}{-22pt}{10pt}
\titlespacing{\section}{12.5mm}{10pt}{10pt}
\titlespacing{\subsection}{12.5mm}{10pt}{10pt}

% ---------- Таблицы --------------------------------------------------
\usepackage{threeparttable}
\usepackage{longtable} % Длинные таблицы
\usepackage{array}     % Для расширенных возможностей работы с таблицами
% \usepackage{threeparttable} %  для создания таблиц с примечаниями, где примечания автоматически подгоняются под ширину таблицы

% Таблица справа (Подпись таблицы)
\usepackage{caption}
%\captionsetup[table]{justification=raggedleft,singlelinecheck=off}
\captionsetup[longtable]{justification=raggedright, singlelinecheck=off}

\usepackage{multirow} % для создания ячеек в таблицах, которые объединяют несколько строк
% для создания многострочных ячеек в таблицах
\newcommand{\specialcell}[2][c]{%
	\begin{tabular}[#1]{@{}c@{}}#2\end{tabular}}
% ---------------------------------------------------------------------
% ---------- Списки----------------------------------------------------
\usepackage{enumerate} % Расширяет возможности нумерованных списков

% переопределение маркеров списка
\usepackage{enumitem}
\def\labelitemi{---}
\setlist[itemize]{leftmargin=1.25cm, itemindent=0.65cm}
\setlist[enumerate]{leftmargin=1.25cm, itemindent=0.55cm}

% нумерованные списки со скобками
\setenumerate[1]{label={ \arabic*)}}
% ---------------------------------------------------------------------
% ---------- Картинки -------------------------------------------------
% Рисунок 1 – Пример изображения.
\captionsetup{labelsep=endash}
\captionsetup[figure]{name={Рисунок}, justification=centering}

% Создание рисунков
\usepackage{graphicx}
\newcommand{\imgScale}[3] {
	\begin{figure}[h!]
		\center{\includegraphics[scale=#1]{img/#2}} % height
		\caption{#3}
		\label{img:#2}
	\end{figure}
}

\newcommand{\imgh}[3] {
\begin{figure}[h!]
    \center{\includegraphics[height=#1]{img/#2}}
    \caption{#3}
    \label{img:#2}
\end{figure}
}
\newcommand{\imgw}[3] {
\begin{figure}[h!]
    \center{\includegraphics[width=#1]{img/#2}}
    \caption{#3}
    \label{img:#2}
\end{figure}
}
\newcommand{\boximg}[3] {
\begin{figure}[h]
    \center{\fbox{\includegraphics[height=#1]{img/#2}}}
    \caption{#3}
    \label{img:#2}
\end{figure}
}

% ---------------------------------------------------------------------
% ---------- Листинг --------------------------------------------------
\usepackage{listings}
\usepackage{xcolor}
\lstset{ %
	language=caml,                 % выбор языка для подсветки (здесь это С)
	basicstyle=\small\ttfamily, % размер и начертание шрифта для подсветки кода
	numbers=none,               % где поставить нумерацию строк (слева\справа)
%	numberstyle=\tiny,           % размер шрифта для номеров строк
%	stepnumber=1,                   % размер шага между двумя номерами строк
	%	numbersep=5pt,                % как далеко отстоят номера строк от подсвечиваемого кода
	showspaces=false,            % показывать или нет пробелы специальными отступами
	showstringspaces=false,      % показывать или нет пробелы в строках
	showtabs=false,             % показывать или нет табуляцию в строках
	frame=single,              % рисовать рамку вокруг кода
	tabsize=2,                 % размер табуляции по умолчанию равен 2 пробелам
	captionpos=t,              % позиция заголовка вверху [t] или внизу [b] 
	breaklines=true,           % автоматически переносить строки (да\нет)
	breakatwhitespace=false, % переносить строки только если есть пробел
	escapeinside={\#*}{*)},   % если нужно добавить комментарии в коде
	abovecaptionskip=-5pt
}

% Возможность использовать $ в листингах
\newcommand{\dlr}{\mbox{\textdollar}}
% ---------------------------------------------------------------------

% оформление списка источников
\makeatletter
\renewcommand{\@biblabel}[1]{#1.}
\makeatother
 
% Ссылки в pdf
 \usepackage[unicode,pdftex]{hyperref}
 \hypersetup{hidelinks}  % Скрывает визуальное оформление всех ссылок
 
 \usepackage{cleveref}



