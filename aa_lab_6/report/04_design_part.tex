\chapter{Конструкторская часть}

В этом разделе представлены схемы муравьиного алгоритма и алгоритма, основанного на полном переборе.

\section{Описание алгоритмов}
В данной работе рассматривается поставка задачи коммивояжера с использованием неориентированного графа, описывающего расстояния между городами России от Калининграда до Владивостока, и производится поиск гамильтонова цикла.

На рисунках~\ref{img:all_combs.pdf}-\ref{img:ant_alg_2.pdf} представлены схемы алгоритма, основанного на полном переборе и муравьиного алгоритма.

\FloatBarrier
\imgh{0.85\textheight}{all_combs.pdf}{Алгоритм основанный на полном переборе}
\FloatBarrier
\imgh{0.95\textheight}{ant_alg_1.pdf}{Муравьиный алгоритм (начало)}
\FloatBarrier
\imgh{0.55\textheight}{ant_alg_2.pdf}{Муравьиный алгоритм (конец)}
\FloatBarrier

\section*{Вывод}

В данном разделе были представлены схемы муравьиного алгоритма и алгоритма, основанного на полном переборе.
\clearpage
