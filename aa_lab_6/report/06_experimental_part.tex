\chapter{Исследовательская часть}

Цель исследования -- сравнительный анализ методов решения задачи коммивояжера

\section{Технические характеристики}
Технические характеристики устройства, на котором выполнялось тестирование:
\begin{itemize}
	\item {Операционная система – Майкрософт Windows 11 Домашняя для одного языка; Версия - 10.0.22631;}
	\item {Установленная оперативная память (RAM) - 16,0 ГБ;}
	\item {Процессор - AMD Ryzen 7 5800H with Radeon Graphics, 3201 МГц, ядер: 8, логических процессоров: 16;}
\end{itemize}

При проведении замеров времени ноутбук был включен в сеть электропитания, а не работал от аккумулятора, и были запущены только встроенное приложение окружения и система замеров времени.

\section{Время выполнения алгоритмов}
Для замера процессорного времени использовалась функция $process\_time\_ns(...)$ из библиотеки $time$\cite{python-time}).

Замеры времени проводились по принципу: для одних входных данных проводилось 5 замеров и если относительная стандартная ошибка среднего (rse) была >= 5\%, то для этих данных замеры продолжались, в таблицу заносилось среднее значение.

Был получен график зависимости времени работы алгоритмов, полного перебора и муравьиного,  от количества узлов в графе~\ref{img:timegr}.
\FloatBarrier
\imgw{0.9\textwidth}{timegr}{Зависимости времени работы алгоритмов, полного перебора и муравьиного,  от количества узлов в графе}
\FloatBarrier

По результатам замеров времени можно сделать вывод о том, что при размерах матрицы больших 10 муравьиный алгоритм работает быстрее, чем алгоритм использующий метод полного перебора.

\section{Результаты параметризации}

Целью параметризации -- определение параметры, при которых задача выбранного класса данных решается с наилучшим результатом.

Параметризация муравьиного алгоритма проводилась по параметрам
\begin{itemize}
 	\item коэффициент жадности алгоритма,  $a \in {0.1, 0.25, 0.5, 0.75, 0.9}$;
 	\item коэффициент испарения феромона,  $p \in {0.1, 0.25, 0.5, 0.75, 0.9}$;
 	\item количество дней, рассматриваемых в алгоритме,  $t_max \in {5, 10, 50, 100, 200}$;
\end{itemize}

Был рассмотрен один класс данных, который состоял из 3 графов~\ref{graph1}-\ref{graph3} представленных матрицами смежности размерностью в 10 элементов. Для вычисления расстояния между городами использовалась библиотека $GeoPy$\cite{GeoPy}.

\begin{equation}
	\label{graph1}
	G_{1} = \begin{pmatrix}
		0 & 2712 & 1382 & 1561 & 1006 & 741 & 5846 & 4112 & 2092 & 1109 \\
		2712 & 0 & 1521 & 4005 & 2947 & 3272 & 3671 & 2835 & 1797 & 1896 \\
		1382 & 1521 & 0 & 2491 & 1426 & 1797 & 5078 & 3774 & 1939 & 1201 \\
		1561 & 4005 & 2491 & 0 & 1091 & 822 & 7378 & 5671 & 3653 & 2664 \\
		1006 & 2947 & 1426 & 1091 & 0 & 654 & 6441 & 4911 & 2901 & 1903 \\
		741 & 3272 & 1797 & 822 & 654 & 0 & 6562 & 4853 & 2832 & 1842 \\
		5846 & 3671 & 5078 & 7378 & 6441 & 6562 & 0 & 2109 & 3825 & 4741 \\
		4112 & 2835 & 3774 & 5671 & 4911 & 4853 & 2109 & 0 & 2025 & 3029 \\
		2092 & 1797 & 1939 & 3653 & 2901 & 2832 & 3825 & 2025 & 0 & 1005 \\
		1109 & 1896 & 1201 & 2664 & 1903 & 1842 & 4741 & 3029 & 1005 & 0 \\
	\end{pmatrix}
\end{equation}
Граф 1~(\ref{graph1}) описывает расстояние в километрах между городами: Барнаул, Екатеринбург, Калининград, Москва, Санкт-Петербург, Владивосток, Якутск, Норильск, Воркута.

\begin{equation}
	\label{graph2}
	G_{2} = \begin{pmatrix}
		0 & 2005 & 6837 & 5312 & 2869 & 5759 & 6999 & 1181 & 1699 & 1993 \\
		2005 & 0 & 5166 & 3804 & 1197 & 3894 & 5468 & 1017 & 1922 & 773 \\
		6837 & 5166 & 0 & 1582 & 4029 & 1544 & 645 & 6168 & 7041 & 4865 \\
		5312 & 3804 & 1582 & 0 & 2614 & 1478 & 1694 & 4759 & 5725 & 3385 \\
		2869 & 1197 & 4029 & 2614 & 0 & 2893 & 4290 & 2151 & 3119 & 881 \\
		5759 & 3894 & 1544 & 1478 & 2893 & 0 & 2109 & 4911 & 5671 & 3774 \\
		6999 & 5468 & 645 & 1694 & 4290 & 2109 & 0 & 6441 & 7378 & 5078 \\
		1181 & 1017 & 6168 & 4759 & 2151 & 4911 & 6441 & 0 & 1091 & 1426 \\
		1699 & 1922 & 7041 & 5725 & 3119 & 5671 & 7378 & 1091 & 0 & 2491 \\
		1993 & 773 & 4865 & 3385 & 881 & 3774 & 5078 & 1426 & 2491 & 0 \\
	\end{pmatrix}
\end{equation}
Граф 2~(\ref{graph2}) описывает расстояние в километрах между городами: Барнаул, Краснодар, Сыктывкар, Хабаровск, Чита, Сургут, Якутск, Владивосток, Москва, Калининград, Екатеринбург.

\begin{equation}
	\label{graph3}
	G_{3} = \begin{pmatrix}
		0 & 851 & 2049 & 2737 & 3537 & 4526 & 4262 & 4710 & 2290 & 5225 \\
		851 & 0 & 1233 & 1957 & 2690 & 3677 & 3411 & 3899 & 3109 & 4376 \\
		2049 & 1233 & 0 & 1477 & 1531 & 2524 & 2320 & 2667 & 4334 & 3313 \\
		2737 & 1957 & 1477 & 0 & 1518 & 2240 & 1857 & 2899 & 4741 & 2664 \\
		3537 & 2690 & 1531 & 1518 & 0 & 994 & 824 & 1392 & 5788 & 1810 \\
		4526 & 3677 & 2524 & 2240 & 994 & 0 & 414 & 972 & 6761 & 922 \\
		4262 & 3411 & 2320 & 1857 & 824 & 414 & 0 & 1332 & 6461 & 993 \\
		4710 & 3899 & 2667 & 2899 & 1392 & 972 & 1332 & 0 & 6999 & 1699 \\
		2290 & 3109 & 4334 & 4741 & 5788 & 6761 & 6461 & 6999 & 0 & 7378 \\
		5225 & 4376 & 3313 & 2664 & 1810 & 922 & 993 & 1699 & 7378 & 0 \\
	\end{pmatrix}
\end{equation}
Граф 3~(\ref{graph3}) описывает расстояние в километрах между городами: Иркутск, Красноярск, Омск, Воркута, Казань, Брянск, Тверь, Краснодар, Владивосток, Калининград.

Таблица значений параметризации представлена в приложении А.

По результатам параметризации было определено несколько наилучших наборов параметров~\ref{tbl:ress1}-\ref{tbl:ress3}:

\FloatBarrier
\begin{longtable}{|
		>{\raggedleft\arraybackslash}m{.1\textwidth - 2\tabcolsep}|
		>{\raggedleft\arraybackslash}m{.1\textwidth - 2\tabcolsep}|
		>{\raggedleft\arraybackslash}m{.1\textwidth - 2\tabcolsep}|
		>{\raggedleft\arraybackslash}m{.2\textwidth - 2\tabcolsep}|
		>{\raggedleft\arraybackslash}m{.25\textwidth - 2\tabcolsep}|
		>{\raggedleft\arraybackslash}m{.25\textwidth - 2\tabcolsep}|
	}
	\caption{Результаты параметризации муравьиного алгоритма для графа 1 (фрагмент)}\label{tbl:ress1} \\\hline
	a & p & t\_max &  max\_diff & avg\_diff & mid\_diff \\ \hline
	\endfirsthead
	\caption*{Продолжение таблицы~\ref{tbl:ress1} } \\ \hline
	a & p & t\_max &  max\_diff & avg\_diff & mid\_diff \\ \hline
	\endhead
	\hline
	\endfoot
	\hline
	0.90 & 0.50 & 200 & 383.00 & 114.90 & 0.00 \\ \hline
	0.90 & 0.25 & 200 & 383.00 & 38.30 & 0.00 \\ \hline
	0.75 & 0.50 & 200 & 1133.00 & 419.90 & 191.50 \\ \hline
	0.75 & 0.75 & 200 & 750.00 & 381.40 & 383.00 \\ \hline
	0.90 & 0.10 & 200 & 383.00 & 38.30 & 0.00 \\ \hline
\end{longtable}
\FloatBarrier
\FloatBarrier
\begin{longtable}{|
		>{\raggedleft\arraybackslash}m{.1\textwidth - 2\tabcolsep}|
		>{\raggedleft\arraybackslash}m{.1\textwidth - 2\tabcolsep}|
		>{\raggedleft\arraybackslash}m{.1\textwidth - 2\tabcolsep}|
		>{\raggedleft\arraybackslash}m{.2\textwidth - 2\tabcolsep}|
		>{\raggedleft\arraybackslash}m{.25\textwidth - 2\tabcolsep}|
		>{\raggedleft\arraybackslash}m{.25\textwidth - 2\tabcolsep}|
	}
	\caption{Результаты параметризации муравьиного алгоритма для графа 2 (фрагмент)}\label{tbl:ress2} \\\hline
	a & p & t\_max &  max\_diff & avg\_diff & mid\_diff \\ \hline
	\endfirsthead
	\caption*{Продолжение таблицы~\ref{tbl:ress2} } \\ \hline
	a & p & t\_max &  max\_diff & avg\_diff & mid\_diff \\ \hline
	\endhead
	\hline
	\endfoot
	\hline
	0.90 & 0.50 & 200 & 0.00 & 0.00 & 0.00 \\ \hline
	0.90 & 0.25 & 200 & 383.00 & 114.90 & 0.00 \\ \hline
	0.75 & 0.50 & 200 & 897.00 & 472.90 & 383.00 \\ \hline
	0.75 & 0.75 & 200 & 964.00 & 337.30 & 191.50 \\ \hline
	0.90 & 0.10 & 200 & 636.00 & 101.90 & 0.00 \\ \hline
\end{longtable}
\FloatBarrier
\FloatBarrier
\begin{longtable}{|
		>{\raggedleft\arraybackslash}m{.1\textwidth - 2\tabcolsep}|
		>{\raggedleft\arraybackslash}m{.1\textwidth - 2\tabcolsep}|
		>{\raggedleft\arraybackslash}m{.1\textwidth - 2\tabcolsep}|
		>{\raggedleft\arraybackslash}m{.2\textwidth - 2\tabcolsep}|
		>{\raggedleft\arraybackslash}m{.25\textwidth - 2\tabcolsep}|
		>{\raggedleft\arraybackslash}m{.25\textwidth - 2\tabcolsep}|
	}
	\caption{Результаты параметризации муравьиного алгоритма для графа 3 (фрагмент)}\label{tbl:ress3} \\\hline
	a & p & t\_max &  max\_diff & avg\_diff & mid\_diff \\ \hline
	\endfirsthead
	\caption*{Продолжение таблицы~\ref{tbl:ress3} } \\ \hline
	a & p & t\_max &  max\_diff & avg\_diff & mid\_diff \\ \hline
	\endhead
	\hline
	\endfoot
	\hline
	0.90 & 0.50 & 200 & 383.00 & 114.90 & 0.00 \\ \hline
	0.90 & 0.25 & 200 & 383.00 & 38.30 & 0.00 \\ \hline
	0.75 & 0.50 & 200 & 383.00 & 76.60 & 0.00 \\ \hline
	0.75 & 0.75 & 200 & 636.00 & 255.10 & 383.00 \\ \hline
	0.90 & 0.10 & 200 & 750.00 & 113.30 & 0.00 \\ \hline
\end{longtable}
\FloatBarrier


\section{Вывод}
В данном разделе было проведено исследование зависимости времени работы муравьиного алгоритма и алгоритма полного перебора, из которого был сделан вывод о том что при размерах матрицы больших 10 муравьиный алгоритм работает быстрее, чем алгоритм использующий метод полного перебора. Также была проведена параметризация, по результатам которой было определено несколько наилучших наборов параметров для описанного класса данных. 


\clearpage
