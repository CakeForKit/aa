\chapter{Аналитическая часть}

В данном разделе рассматриваются методы решения задачи коммивояжера основанные на полном переборе и на муравьином алгоритме.


\section{Задача коммивояжера}

Задача коммивояжера: на основе заданного графа и информации о расстоянии между его вершинами составить такой путь/цепь (далее <<маршрут>>), чтобы все узлы входили в него по 1 разу и при этом сумма меток дуг/ребер, соединяющих каждую последовательну пару вершин маршрута, была минимальна. Также возможна другая постановка задачи с возвратом в начальные город -- поиск гамельтонова пути.

\section{Метод полного перебора}

Метод полного перебора заключается в том, чтобы рассмотреть все возможные варианты маршрута в рассматриваемом графе. Главным недостатком алгоритма является его ассимптотическая сложность, которая равна (O(n!)), где n~-- это количество узлов в графе. Достоинством данного решения является гарантированное нахождение наилучшего маршрута.


\section{Метод на основе муравьиного алгоритма}

Муравьиный алгоритм~\cite{algos} заключается в поиске решения, основанном на поведении муравьев. Главным достоинством данного алгоритма является более низкая временная сложность, относительно алгоритма основанного на полном переборе. Но для данного метода не гарантированно нахождение наилучшего результа.

За каждые сутки $t \in [1, t_{max}]$ каждый муравей $k \in [1, n]$, где $n$~-- это количество узлов в графе, формирует 1 маршрут, который претендует на решение задачи коммивояжера.

Муравей обладает следующими свойствами:
\begin{itemize}
	\item \textbf{зрение} -- муравей $k$ стоя в вершине $i$ может оценить привлекательность перехода $\eta_{ij}$ из города $i$ в город $j$
	\begin{equation}
		\eta_{ij} = 1 / D_{ij},
	\end{equation}
	где $D_{ij}$ — расстояние между вершинами $i$ и $j$;
	
	
	\item \textbf{память} -- муравей $k$ запоминает посещение в день $t$ города в кортеж посещенных городов $I_{k}(t)$;
	
	\item \textbf{обоняние} -- муравей чует концентрацию феромона $\tau_{ij}(t)$ на дуге $ij$;
	
\end{itemize}

Общий вид алгоритма: 

Для каждого $t \in [1, t_{max}]$
\begin{enumerate}[label={\arabic*)}]
	\item утром муравьи выходят из муравейника, каждый муравей встает в свою вершину (муравьи распределяются по одному в каждом узле графа);
	\item за день $t$ каждый муравей $k$ формирует по одному маршруту;
	\item к закату муравь возвращаются в муравейник и при необходимости обновляется наилучший из найденных маршрутов;
	\item ночью обновляется феромон;
\end{enumerate}

Стоя в день $t$ в вершине $i$ муравей $k$ рассчитывает вероятность перехода в вершину $j$

\begin{equation}
	P_{kij} = \begin{cases}
		\frac{\eta_{ij}^a\tau_{ij}^{1-a}}{\sum_{q=1}^n \eta^a_{iq}\tau^{1-a}_{iq}}, \textrm{$j \notin I_{k}(t)$ } \\
		0, \textrm{$j \in I_{k}(t)$ }
	\end{cases}
\end{equation}
где $a$ -- коэффицент жадности решения.

Формула обновления феромона:
\begin{equation}
	\tau_{ij}(t+1) = \tau_{ij}(t)(1-p) + \Delta \tau_{ij}(t).
\end{equation}
где $р \in (0, 1)$ -- коэффицент испарения феромона.
\begin{equation}
	\Delta \tau_{ij}(t) = \sum_{k=1}^N \Delta \tau_{k,ij}(t),
\end{equation}
\begin{equation}
	\Delta \tau_{k,ij}(t) = \begin{cases}
		\frac{Q}{L_{k}(t)}, \textrm{если муравей $k$ в день $t$ ходил по ребру $ij$} \\
		0, \textrm{иначе}
	\end{cases}
\end{equation}
где $L_{k}(t)$~-- длина маршрута муравья $k$ в день $t$, $Q$~-- дневная квота феромона каждого муравья.

При этом, если $\tau_{ij}(t+1)$ стал меньше некоторой малой константы, то его значение откатывается дл этой малой константы.

\section{Модмификация с элитными муравьями}
Элитный муравей усиливает феромон дополнительной квотой дуги наилучшего маршрута длиной $L_{best}$
\begin{equation}
	\Delta \tau_{ij}(t) = \sum_{k=1}^N \Delta \tau_{k,ij}(t) + \Delta \tau_{э}(t),
\end{equation}
\begin{equation}
	\Delta \tau_{э}(t) = \begin{cases}
		\frac{Q}{L_{best}}, \textrm{если дуга входит в наилучший маршрут} \\
		0, \textrm{иначе}
	\end{cases}
\end{equation}

%\clearpage
\section*{Вывод}
В данном разделе была рассмотрена задача коммивояжера и методы ее решения на основе полного перебора и муравьиного алгоритма.
\clearpage
