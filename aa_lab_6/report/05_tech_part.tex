\chapter{Технологическая часть}

В данном разделе представлены средства реализации, листинги муравьиного алгоритма, алгоритма основанного на полном переборе и функциональные тесты.

\section{Средства реализации}
В данной работе для реализации был выбран язык программирования $Python$\cite{python-city}, так как он удовлетворяет требованиям лабраторной работы: может замерить процессорное время (с помошью функции $process\_time\_ns(...)$ из библиотеки $time$\cite{python-time})

В данной работе для реализации был выбран язык программирования $Python$\cite{python-city}, так как он удовлетворяет требованиям лабраторной работы: поддерживает динамические структуры данных,
такие как массивы, имеет библиотеки $GeoPy$\cite{GeoPy} для вычисления расстояния городами и  $Matplotlib$\cite{matplotlib-lib} для построения графиков и может замерить процессорное время (с помошью функции $process\_time\_ns(...)$ из библиотеки $time$\cite{python-time})


\section{Реализация алгоритмов}
В листингах \ref{lst:lst1}-\ref{lst:lst2} представлены реализации алгоритма основанного на полном переборе и муравьиного алгоритма.

\begin{center}
	\captionsetup{justification=raggedright,singlelinecheck=off}
	\renewcommand{\lstlistingname}{Листинг}
	\begin{lstlisting}[label=lst:lst1,caption=Реализация алгоритма основанного на полном переборе]
def all_combs(graph : CitiesMap):
	lenpath = len(graph)
	inds = [i for i in range(1, lenpath)]
	
	minDist = float('inf')
	goodPath = []
	for path in permutations(inds):
		path = [0] + list(path)
		dist = 0
		for i in range(lenpath - 1):
			dist += graph[path[i]][path[i + 1]]
		dist += graph[path[0]][path[-1]]
		
		if dist < minDist:
			minDist = dist
			goodPath = path[:]
	
	return (minDist, goodPath)
	\end{lstlisting}
\end{center}

\begin{center}
	\captionsetup{justification=raggedright,singlelinecheck=off}
	\renewcommand{\lstlistingname}{Листинг}
	\begin{lstlisting}[label=lst:lst2, caption=Реализация муравьиного алгоритма]
class Ant:
	def __init__(self, istart, cmap: CitiesMap, pheromones, greedy):
		self.path = self.generatePath(istart, cmap, pheromones, greedy)
		self.lenPath = self.calcLenPath(cmap, self.path)
	
	def getPath(self):
		return self.path[:]
	
	def getLenPath(self):
		return self.lenPath

	def generatePath(self, istart, cmap: CitiesMap, pheromones, greedy, printing=False):
		path = [istart]
		ncities = len(cmap)
		while len(path) != ncities:
			probPaths = [0] * ncities
			for i in range(ncities):
				if i in path:
					probPaths[i] = 0
				else:
					ist = path[-1]
					probPaths[i] = cmap.attractiveness(ist, i) ** greedy * \
						pheromones[ist][i] ** (1 - greedy)
			sumProbPaths = sum(probPaths)
			for i in range(ncities):
				probPaths[i] /= sumProbPaths
			
			r = random.random()
			last = 0
			lines = []
			for i in range(len(probPaths)):
				lines.append((last, last + probPaths[i]))
				last = last + probPaths[i]

			i = 0
			while not( lines[i][0] <= r < lines[i][1] ):
				i += 1
			ind = i


			path.append(ind)
		return path

	def calcLenPath(self, cmap, path):
		lenPath = 0
		for i in range(len(cmap) - 1):
			lenPath += cmap[path[i]][path[i + 1]]
		lenPath += cmap[path[-1]][path[0]]
		return lenPath

class Pheromones:
	min_ph = 0.01
	def __init__(self, size):
		self.phmap = []
		for i in range(size):
			self.phmap.append([1 for _ in range(size)])
			self.phmap[i][i] = 0
	
	def __getitem__(self, item):
		return self.phmap[item]

	def update(self, ants, lenBestPath, cmap: CitiesMap, evaporation, printing=False):
		deltaMap = [[0 for i in range(len(cmap))] for _ in range(len(cmap))]
		
		for a in ants:
			path = a.getPath()
			delta = cmap.day_pheromone / a.getLenPath()
			for i in range(len(path) - 1):
				deltaMap[path[i]][path[i + 1]] += self.phmap[path[i]][path[i + 1]] * (1 - evaporation) + delta
				deltaMap[path[i + 1]][path[i]] = deltaMap[path[i]][path[i + 1]]


			deltaMap[path[0]][path[-1]] += self.phmap[path[i]][path[i + 1]] * (1 - evaporation) + delta
			deltaMap[path[-1]][path[0]] = deltaMap[path[0]][path[-1]]

		for i in range(len(cmap)):
			for j in range(i):
				self.phmap[i][j] = max(self.min_ph, deltaMap[i][j])
				self.phmap[j][i] = self.phmap[i][j]

def ant_alg(cmap: CitiesMap, greedy, evaporation, t_max):
	bestPath = []
	lenBestPath = float('inf')
	pheromones = Pheromones(len(cmap))
	
	for t in range(t_max):
		ants = [Ant(i, cmap, pheromones, greedy) for i in range(len(cmap))]
		for a in ants:
			if a.getLenPath() < lenBestPath:
				bestPath = a.getPath()
				lenBestPath = a.getLenPath()
		pheromones.update(ants, lenBestPath, cmap, evaporation)

	return (lenBestPath, bestPath)
	\end{lstlisting}
\end{center}


\section{Классы эквивалентности тестирования}

Для тестирования были выделены следующие классы тестирования:
\begin{enumerate}
	\item {Граф пустой};
	\item {Граф с 1 вершиной};
	\item {Граф с 2 вершинами};
	\item {Граф с 3 вершинами};
\end{enumerate}

\section{Функциональные тесты}
В таблице \ref{tbl:functional_test} приведены функциональные тесты для алгоритмов умножения матриц.

Все тесты пройдены успешно.

\FloatBarrier
\begin{longtable}{|
		>{\centering\arraybackslash}m{.4\textwidth - 2\tabcolsep}|
		>{\centering\arraybackslash}m{.3\textwidth - 2\tabcolsep}|
		>{\centering\arraybackslash}m{.3\textwidth - 2\tabcolsep}|
	}
	\caption{Функциональные тесты}\label{tbl:timeData} \\\hline
	Граф представленный матрицей смежности & Ожидаемая длина маршрута & Результат программы \\ \hline
	\endfirsthead
	\caption*{Продолжение таблицы~\ref{tbl:timeData} } \\\hline
	Граф представленный матрицей смежности & Ожидаемая длина маршрута & Результат программы \\ \hline              
	\endhead
	\endfoot
	$\begin{pmatrix}
		 & 
	\end{pmatrix}$ & 0 & 0 \\ \hline
	$\begin{pmatrix}
		10 
	\end{pmatrix}$ & 0 & 0 \\ \hline
	$\begin{pmatrix}
		0 & 2 \\
		2 & 0 
	\end{pmatrix}$ & 2 & 2 \\ \hline
	$\begin{pmatrix}
		0 & 1 & 2 \\
		1 & 0 & 3 \\
		2 & 3 & 0
	\end{pmatrix}$ & 6 & 6 \\ \hline
\end{longtable}
\FloatBarrier


\section*{Вывод}

В данном разеде были представлены средства реализации, листинг кода и функциональные тесты.

\clearpage
