\ssr{ЗАКЛЮЧЕНИЕ}

В ходе курсовой работы была выполнена поставленная цель, которая заключалась в сравнительном анализе методов решения задачи коммивояжера.

В процессе выполнения данной работы были выполнены все задачи:
\begin{enumerate}
	\item описать методы решения задачи коммивояжера, основанные на полном переборе и на муравьином алгоритме;
	\item выделить преимущества и недостатки рассматриваемых методов;
	\item описать схему алгоритмов решения задачи коммивояжера;
	\item реализовать алгоритмы решения задачи коммивояжера;
	\item выполнить параметризацию метода основанного на муравьином алгоритме;
	\item провести сравнительный анализ методов решения задачи коммивояжера.
\end{enumerate}

Основываясь на проведенном исследовании зависимости времени работы алгоритмов и параметризации муравьиного алгоритма был сделан вывод о том, что при размерах матрицы больших 10 муравьиный алгоритм работает быстрее, также было определено несколько наилучших наборов параметров для описанного класса данных.