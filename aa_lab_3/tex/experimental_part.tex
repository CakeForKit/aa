\chapter{Исследовательская часть}

Цель исследования -- сравнительный анализ реализованных алгоритмов по трудоемкости.


\section{Технические характеристики}
Технические характеристики устройства, на котором выполнялось тестирование:
\begin{itemize}
	\item {Операционная система – Майкрософт Windows 11 Домашняя для одного языка; Версия -- 10.0.22631; Сборка -- 22631;}
	\item {Установленная оперативная память (RAM) -- 16,0 ГБ;}
	\item {Процессор -- AMD Ryzen 7 5800H with Radeon Graphics, 3201 МГц, ядер: 8, логических процессоров: 16;}
\end{itemize}

\section{Время выполнения алгоритмов}
Трудоемкость алгоритмов определялась в терминаз числа сравнений, которые понадобились на нахождение искомого элемента.

Согласно варианту, была посчитана длина массива по следующей формуле:

\begin{align}
	n = 
	\begin{cases}
		X \mod 1000, \text{ если} \left( \frac{X}{4} \mod 10 \right) = 0, \\
		\left( \frac{X}{4} \mod 10 \right) \times (X \mod 10) + \left( \frac{X}{2} \mod 10 \right), \text{ иначе}
	\end{cases}
	\tag{4.1} \label{eq:formula}
\end{align}

\text{где} \ X = 8119

Длина массив = 1014 элементов

Были полученны гистограммы \ref{img:graphCompleteSearch}-\ref{img:bin_search_sort} для алгоритмов поиска элемента в массиве


\imgScale{0.5}{graphCompleteSearch}{Алгоритм поиска элемента в массиве полным перебором}
\imgScale{0.5}{bin_search}{Алгоритма поиска элемента в массиве бинарным поиском}
\clearpage
\imgScale{0.5}{bin_search_sort}{Алгоритма поиска элемента в массиве бинарным поиском в порядке возрастания количества сранений}



\section{Вывод}
Алгоритм бинарного поиска требует значительно меньше сравнений чем алгоритм использующий полный перебор. Для массива длиной 1014 максимальное количество сравнений в алгоритме бинарного поиска = 30, а в алгоритме полного перебора = 2000. 

Но при этом в случаях когда искомых элемент находится в начале массива (индекс до 30) алгоритм полного перебора требует меньше сравнений.

\clearpage
