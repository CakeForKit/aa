\chapter{Технологическая часть}
В данном разделе будут приведены средства реализации, листинг кода и функциональные тесты.

\section{Средства реализации}
В данной работе для реализации был выбран язык программирования $Python$\cite{python-city}, так как он удовлетворяет требованиям лабраторной работы: поддерживает динамические структуры данных,
такие как массивы и имеет библиотеку $Matplotlib$\cite{matplotlib-lib} для построения графиков.


\section{Реализация алгоритмов}
В листингах \ref{lst:completeSearch}-\ref{lst:binSearch} представлены реализации алгоритмов поиска элемента в массиве.

\begin{center}
	\captionsetup{justification=raggedright,singlelinecheck=off}
	\renewcommand{\lstlistingname}{Листинг}
	\begin{lstlisting}[label=lst:completeSearch,caption=Алгоритм поиска элемента в массиве полным перебором]
		def completeSearch(arr, x, printing=False):
			count_cmp = 0
			for i in range(len(arr)):
				count_cmp += 2
				if arr[i] == x:
					return i, count_cmp
			return -1, count_cmp
	\end{lstlisting}
\end{center}

\begin{center}
	\captionsetup{justification=raggedright,singlelinecheck=off}
	\renewcommand{\lstlistingname}{Листинг}
	\begin{lstlisting}[label=lst:binSearch, caption=Алгоритма поиска элемента в массиве бинарным поиском]
	def binSearch(arr, x, printing=False):
		arr.sort()
		n = len(arr)
		l, r = 0, n - 1
		m = n // 2
		
		count_cmp = 0
		while arr[m] != x and l <= r:
			count_cmp += 3
			if x > arr[m]:
				l = m + 1
			else:
				r = m - 1
			m = (r + l) // 2
	count_cmp += 3
	if l > r:
		return -1, count_cmp
	else:
		return m, count_cmp
	\end{lstlisting}
\end{center}

\section{Классы эквивалентности тестирования}

Для тестирования были выделены следующие классы тестирования:
\begin{enumerate}
	\item {Пустой массив};
	\item {Искомый элемент -- первый};
	\item {Искомый элемент -- последний};
	\item {Искомый элемент -- в середине массива};
	\item {Искомого элемена -- нет};
\end{enumerate}

\section{Функциональные тесты}
В таблице \ref{tbl:functional_test} приведены функциональные тесты для алгоритмов поиска элемента $x$ в массиве $arr = [4, 5, 12, 13, 13, 18, 22, 26, 30, 35]$. 

Все тесты пройдены успешно.

\FloatBarrier
\begin{table}[h]
	\begin{center}
		\begin{threeparttable}
			\captionsetup{justification=raggedright,singlelinecheck=off}
			\caption{\label{tbl:functional_test} Функциональные тесты}
			\begin{tabular}{|c|c|c|c|c|}
				\hline
				& & & \multicolumn{2}{c|}{Результат} \\
				\hline
				№ & x & Массив & Поный перебор & Бинарный поиск \\
				\hline
				1& 4 & "Пустой массив" & -1 & -1 \\
				\hline
				2& 4 & arr & 0 & 0 \\
				\hline
				3& 5 & arr &4&4 \\
				\hline
				4& 35 & arr &9&9 \\
				\hline
				5& 18 & arr &5&5 \\
				\hline
			\end{tabular}
		\end{threeparttable}
	\end{center}
\end{table}
\FloatBarrier


\section*{Вывод}

Были представлены листинги всех описанных ренее алгоритмов поиска элемента в массиве и их тесты.

\clearpage
