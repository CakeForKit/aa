\chapter{Аналитическая часть}

В данной работе будут рассмотрены два стандартных подхода поиска элемента в массиве: полный перебор и бинарный поиск.

\section{Алгоритм поиска элемента в массиве полным перебором}

Поиск полным перебором предполагает сравнение искомого элемента с каждым элементом массива. Возможно (N + 1) случаев: элемент не найден и N возможных случаев расположения жлемента в массиве.

Лучший случай: за одно сравнение элемент найден в начале массива.

Худших случаев два: за N сравнений либо элемент не найден, либо ключ найден на последнем сравнении.

Пусть на старте алгоритм поиска затрачивает $k_{0}$ операций, а при каждом сравнении $k_{1}$ сравнений. Тогда в лучшем случае будет затрачено $k_{0} + k_{1}$ операций. В случае, если ключ будет найден на второй позиции, будет затрачено $k_{0} + 2k_{1}$, на последней позиции -  $k_{0} + Nk_{1}$ операций, столько же если ключ не будет найден вовсе.

%Трудоемкость в среднем:
%\begin{equation}
%	f_{mid} = \displaystyle\sum_{i\in\Omega} p_{i} * S_{i}
%\end{equation}
%\begin{equation}
%	f_{mid} = (k_{0} + k_{1}) * \frac{1}{N + 1} + (k_{0} + 2k_{1}) * \frac{1}{N + 1} + ... + (k_{0} + Nk_{1}) * \frac{1}{N + 1}
%\end{equation}
%\begin{equation}
%	\label{eq:D}
%	f_{mid} = k_{0} + k_{1}(1 + \frac{N}{2} - \frac{1}{N + 1})
%\end{equation}
%где $\Omega$ - множество всех возможных случаев.

%Средний слйчай - это сумма вероятностей случаев, взвешенных вероятностями этих случаев.

Пусть $S_{1}$ и $S_{2}$ - две строки, длиной \textit{M} и \textit{N} соответственно, над некоторым алфавитом, тогда расстояние Левенштейна d($s_{1}$, $s_{2}$) можно подсчитать по следующей рекуррентной формуле d($s_{1}$, $s_{2}$) = D(M, N):

\section{Алгоритма поиска элемента в массиве бинарным поиском}
При двоичном поиске обход можно представить деревом, поэтому трудоемкость в худшем случае составит $\log_2 N$ (в худшем случае нужно спуститься по двоичному дереве от корня до листа)

Скорость роста функции $\log_2 N$ меньше чем у N


\section*{Вывод}
В данном разделе были теоретически разобраны два стандартных подхода поиска элемента в массиве: полный перебор и бинарный поиск.


\clearpage
