\ssr{ЗАКЛЮЧЕНИЕ}


В ходе лабораторной работы была выполнена поставленнная цель, которая заключалась в сравнительном анализе алгоритмов поиска элемента в массиве, используя полный перебор и бинарный поиск.

Были выполнены следующие задачи:
\begin{enumerate}
	\item Изучить алгоритмы поиска элемента в массиве, используя полный перебор и бинарный поиск;
	\item Реализовать алгоритмы:
	\begin{itemize}[label=---]
		\item поиска элемента в массиве полным перебором;
		\item поиска элемента в массиве бинарным поиском;
	\end{itemize}
	\item Провести сравнительный анализ по трудоемкости работы алгоритмов;
\end{enumerate}


Основываясь на проведенном исследовании можно сделать следующий вывод.
Алгоритм бинарного поиска треубет значительно меньше сравнений чем алгоритм использующий полный перебор. Для массива длиной 1014 максимальное количество сравнений в алгоритме бинарного поиска = 30, а в аглритме полного перебора = 2000.

Но при этом в случаях когда искомых элемент находится в начале массива алгоритм полного перебора требует меньше сравнений.
