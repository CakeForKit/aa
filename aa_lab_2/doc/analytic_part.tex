\chapter{Аналитическая часть}

В данной работе будут рассмотрены алгоритмы умножения матриц: стандартный и Винограда.

\section{Стандартный алгоритм умножения матриц}

Пусть даны две матрицы
Пусть даны две матрицы $A$ и $B$ размерности $l \times m$ и $m \times n$ соответственно:

\begin{equation}
	A_{lm} = \begin{pmatrix}
		a_{11} & a_{12} & \ldots & a_{1m}\\
		a_{21} & a_{22} & \ldots & a_{2m}\\
		\vdots & \vdots & \ddots & \vdots\\
		a_{l1} & a_{l2} & \ldots & a_{lm}
	\end{pmatrix},
	\quad
	B_{mn} = \begin{pmatrix}
		b_{11} & b_{12} & \ldots & b_{1n}\\
		b_{21} & b_{22} & \ldots & b_{2n}\\
		\vdots & \vdots & \ddots & \vdots\\
		b_{m1} & b_{m2} & \ldots & b_{mn}
	\end{pmatrix},
\end{equation}

тогда матрица $C$ размерностю $l \times n$
\begin{equation}
	C_{ln} = \begin{pmatrix}
		c_{11} & c_{12} & \ldots & c_{1n}\\
		c_{21} & c_{22} & \ldots & c_{2n}\\
		\vdots & \vdots & \ddots & \vdots\\
		c_{l1} & c_{l2} & \ldots & c_{ln}
	\end{pmatrix},
\end{equation}

где
\begin{equation}
	\label{eq:M}
	c_{ij} =
	\sum_{r=1}^{m} a_{ir}b_{rj} \quad (i=\overline{1,l}; j=\overline{1,n})
\end{equation}

будет называться произведением матриц $A$ и $B$.

Операция умножения двух матриц выполнима только в том случае, если число столбцов в первом сомножителе равно числу строк во втором.



\section{Алгоритм Винограда}
Ключевая идея алгоритма Винограда -- снизить долю операций умножения, заменив их операциями сложения, которые являются более эффективными по времени.

Пусть есть два вектора $V = (v_1, v_2, v_3, v_4)$ и $W = (w_1, w_2, w_3, w_4)$.

Их скалярное произведение равно (\ref{for:new1}): 
\begin{equation}
	\label{for:new1}
	V \cdot W = v_1w_1 + v_2w_2 + v_3w_3 + v_4w_4
\end{equation}
что эквивалентно (\ref{for:new}):
\begin{equation}
\begin{aligned}
	\label{for:new}
	V \cdot W &= (v_1 + w_2)(v_2 + w_1) + (v_3 + w_4)(v_4 + w_3) - \\
	&- v_1v_2 - v_3v_4 - w_1w_2 - w_3w_4.
\end{aligned}
\end{equation}

Выражения $- v_1v_2 - v_3v_4$ и $- w_1w_2 - w_3w_4$ можно вычислить заранее и использовать повторно при умножении строки $V$ матрицы $A$ на все столбцы $W$ матрицы $B$. 

Это позволит выполнить меньшее количество операций умножения: 2 умножения и 5 сложений, вместо 4 умножений и 4 сложений. Но при нечетном значении размера матрицы нужно дополнительно добавить произведения крайних элементов соответствующих строк и столбцов.

Операция сложения выполняется быстрее, поэтому на практике алгоритм должен работать быстрее обычного алгоритма перемножения матриц.

\section*{Вывод}
В данном разделе были теоретически разобраны два алгоритмы умножения матриц: стандартного и Винограда.


\clearpage
