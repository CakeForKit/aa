\chapter{Исследовательская часть}

Цель исследования -- сравнительный анализ реализованных алгоритмов по трудоемкости.


\section{Технические характеристики}
\begin{itemize}
	\item {Операционная система – Майкрософт Windows 11 Домашняя для одного языка; Версия -- 10.0.22631; Сборка -- 22631;}
	\item {Установленная оперативная память (RAM) -- 16,0 ГБ;}
	\item {Процессор -- AMD Ryzen 7 5800H with Radeon Graphics, 3201 МГц, ядер: 8, логических процессоров: 16;}
	\item {Микроконтроллер -- STM32F303 \cite{plata};}
	
\end{itemize}

\section{Время выполнения алгоритмов}
Замеры времени работы алгоритмов проводились на плате и для этого использовалась функция ticks\_ms(...) из библиотеки time на MicroPython\cite{python-time}.

Замеры проводились для четных размеров матриц от 2 до 40 по 10 раз на различных входных матрицах. А также -- для нечетных размеров матриц от 3 до 39 по 10 раз на различных входных данных.

Были полученны графики зависимости времени работы плгоритма от размеров квадратных матриц \ref{img:nechet}-\ref{img:chet} для алгоритмов умножения матриц.


\imgScale{0.8}{nechet}{Сравнение времени работы алгоритмов умножения матриц нечетного размера}
\imgScale{0.8}{chet}{Сравнение времени работы алгоритмов матриц четного размера}
\imgScale{0.8}{nechet_2}{Сравнение времени работы алгоритмов умножения матриц нечетного размера для разных реализаций алгоритма винограда}
\imgScale{0.8}{chet_2}{Сравнение времени работы алгоритмов матриц четного размера для разных реализаций алгоритма винограда}
\clearpage



\section{Вывод}
Из проведённых замеров можно сделать следующие выводы:
\begin{itemize}
	\item{Оптимизированный алгоритм Винограда демонстрирует наилучшие результаты по времени работы на всех тестовых данных, как в расчитанной ранее формула, так и на практике;}
	\item{Как и ожидалось умножение матрицы нечетного размера требует больше времени;}
	\item{На практике стандартный алгоритм умножения матриц работает медленнее, чем неоптимизированный алгоритм Винограда, это можно обьяснять оптимизацией компилятора или неточность введенной модели вычислений.}
\end{itemize}


\clearpage
