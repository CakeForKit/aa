\aasection{Входные и выходные данные}{sec:input-output}
Входными данными программы является txt-файл, состоящий из строк вид <строковый ключ из символов a-zA-Z0-9>: <число>.

Выходными данными программы являются N файлов с именами group\_<i>.<имя исходного файла>.txt, где i -- номер группы, N -- количество логических ядер машины. 
\clearpage

\aasection{Преобразование входных данных в выходные}{sec:demo}
Программа выполняет раскладку чисел в группы (размера N, N — по умолчанию количество логических ядер машины) так, чтобы разница максимальной и минимальной среди сумм чисел каждой группы была наименьшей, и записывает в N файлов с именами group\_<i>.<имя исходного файла>.txt, где i — номер группы, каждую группу в формате

\begin{description}
	\item <сумма чисел группы>
	\item <ключ 1>: <значение 1>
	\item ...
	\item <ключ N>: <значение N>
\end{description}

Ключи отсортированы в лексикографическом порядке по возрастанию
\clearpage


\aasection{Примеры работы программы}{sec:demo}
На рисунке~\ref{img:log} представлены пример содержимого лог-файла, где $tstart$, $tend$ -- начало и конец обслуживания заявки с номером $request$ выполняемой на потоке $thread$.

\imgScale{1}{log}{Пример содержимого лог-файла}
\FloatBarrier

\clearpage
\aasection{Тестирование}{sec:tests}

В таблице~\ref{tbl:tests} представлены функциональные тесты разработанного алгоритма распределения чисел по группам так, чтобы разница максимальной и минимальной среди сумм чисел каждой группы была наименьшей. Для упрощения определения правильности работы алгоритма было взято N равное 3. 

Все тесты пройдены успешно.


\begin{longtable}{|
		>{\raggedright\arraybackslash}p{.05\textwidth - 2\tabcolsep}|
		>{\raggedright\arraybackslash}p{.15\textwidth - 2\tabcolsep}|
		>{\raggedright\arraybackslash}p{.4\textwidth - 2\tabcolsep}|
		>{\raggedright\arraybackslash}p{.4\textwidth - 2\tabcolsep}|
	}
	\caption{Функциональные тесты}\label{tbl:tests} \\\hline
	№ & Входные данные & Выходные данные & Ожидаемые выходные данные                                          \\\hline
	\endfirsthead
	\caption{Функциональные тесты (продолжение)} \\\hline
	№ & Входные данные & Выходные данные & Ожидаемые выходные данные                                          \\\hline                          
	\endhead
	\endfoot
	1 & 0 5 9 9 8 3 2 1 0 2 & Группа 1: 0 9 2 2 0 (сумма 13) & Группа 1: 0 9 2 2 0 (сумма 13)\\
	& & Группа 2: 5 8	(сумма 13) & Группа 2: 5 8	(сумма 13)\\
	& & Группа 3: 9 3 1 (сумма 13) & Группа 3: 9 3 1 (сумма 13)\\
	\hline
	2 & - & Группа 1:  (сумма 0)  & Группа 1:  (сумма 0)\\
	& & Группа 2:  (сумма 0) & Группа 2:  (сумма 0) \\
	& & Группа 3:  (сумма 0) & Группа 3:  (сумма 0)\\
	\hline
	3 & 1 1 1 & Группа 1: 1   (сумма 1) & Группа 1: 1   (сумма 1) \\
	& & Группа 2: 1   (сумма 1) & Группа 2: 1   (сумма 1)\\
	& & Группа 3: 1  (сумма 1) & Группа 3: 1  (сумма 1)\\
	\hline
\end{longtable}



\clearpage
