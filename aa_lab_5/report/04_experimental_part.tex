\aasection{Описание исследования}{sec:study}

Был выбран язык программирования с++, так как на linux с его помощью можно создавать нативные потоки~\cite{thread-city}. Проводились замеры реального времени работы программы, для этого использовалась функция gettimeofday~\cite{time-city}. Замеры проводились на виртуальной машине Linux с 8~логическими ядрами, и замерялось время для 3 дополнительных рабочих потоков и 1 главного потока--диспетчера.

Было проведено исследование параллельности обработки задач при конвейерном методе. В результате был сформирован лог обработки задач~\ref{tbl:log} и рассчитаны средние временные характеристики~\ref{tbl:times}.

%\imgScale{0.2}{graph}{Зависимость времени отрисовки сцены от числа полигонов на сцене для варьируемого числа рабочих потоков}

\begin{longtable}{|
		>{\raggedright\arraybackslash}p{.2\textwidth - 2\tabcolsep}|
		>{\raggedright\arraybackslash}p{.2\textwidth - 2\tabcolsep}|
		>{\raggedright\arraybackslash}p{.3\textwidth - 2\tabcolsep}|
		>{\raggedright\arraybackslash}p{.3\textwidth - 2\tabcolsep}|
	}
	\caption{Лог}\label{tbl:log} \\\hline
	№ заявки & № потока & Время начала обслуживания заявки, мс & Время конца обслуживания заявки, мс \\\hline
	\endfirsthead
	\caption*{Продолжение таблицы~\ref{tbl:log} } \\\hline
	№ заявки & № потока & Время начала обслуживания заявки, мс & Время конца обслуживания заявки, мс \\\hline                    
	\endhead
	\endfoot
	0 & 1 & 238 & 250 \\\hline
	0 & 2 & 251 & 254 \\\hline
	0 & 3 & 254 & 373 \\\hline
	1 & 1 & 256 & 256 \\\hline
	2 & 1 & 257 & 257 \\\hline
	3 & 1 & 257 & 257 \\\hline
	4 & 1 & 258 & 258 \\\hline
	5 & 1 & 258 & 258 \\\hline
	6 & 1 & 259 & 259 \\\hline
	7 & 1 & 259 & 259 \\\hline
	8 & 1 & 260 & 260 \\\hline
	9 & 1 & 260 & 260 \\\hline
	10 & 1 & 261 & 280 \\\hline
	1 & 2 & 265 & 266 \\\hline
	2 & 2 & 266 & 267 \\\hline
	... & ... & ... & ... \\\hline
	496 & 3 & 36204 & 36270 \\\hline
	497 & 3 & 36271 & 36339 \\\hline
	498 & 3 & 36340 & 36404 \\\hline
	499 & 3 & 36404 & 36471 \\\hline
	
\end{longtable}


\begin{longtable}{|
		>{\raggedright\arraybackslash}p{.7\textwidth - 2\tabcolsep}|
		>{\raggedright\arraybackslash}p{.3\textwidth - 2\tabcolsep}|
	}
	\caption{Средние временные характеристики}\label{tbl:times} \\\hline
	Средние временные характеристики & Время в мс \\\hline
	\endfirsthead
	\caption{Средние временные характеристики (продолжение)} \\\hline
	Средние временные характеристики & Время в мс \\\hline                    
	\endhead
	\endfoot
	Среднее время существования задачи & 18519.166 \\\hline
	Среднее время ожидания задачи в очереди 2 & 321.256 \\\hline
	Среднее время ожидания задачи в очереди 3 & 18124.676 \\\hline
	Среднее время обработки задачи на стадии 1 & 0.354 \\\hline
	Среднее время обработки задачи на стадии 2 & 1.216 \\\hline
	Среднее время обработки задачи на стадии 3 & 71.664 \\\hline
	
\end{longtable}

Из проведённых замеров был сделан вывод о том, что решение задач происходит параллельно, так как временные отметки начала и конца обработки задач пересекаются.



\clearpage
