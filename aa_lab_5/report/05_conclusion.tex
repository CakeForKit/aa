\ssr{ЗАКЛЮЧЕНИЕ}

В ходе лабораторной работы была выполнена поставленная цель, которая заключалась в разработке программного обеспечения, которое будет раскладывать числа в группы  так, чтобы разница максимальной и минимальной среди сумм чисел каждой группы была наименьшей, с использованием конвейерного метода.


Были выполнены следующие задачи:
\begin{enumerate}
	\item разбить поставленную задачу на 3 подзадачи;
	\item реализовать решение каждой из подзадач;
	\item организовать выполнение подзадач по конвейерному принципу с использованием потока-генератора и трех дополнительных потоков;
\end{enumerate}


Основываясь на проведённом исследовании был сделан вывод о том, что решение задач происходит параллельно, так как временные отметки начала и конца обработки задач пересекаются.

