\aasection{Описание исследования}{sec:study}

Было проведено исследование зависимости времени отрисовки сцены от числа полигонов на сцене для варьируемого числа рабочих потоков. 

Был выбран язык программирования с++, так как на linux с его помощью можно создавать нативные потоки~\cite{thread-city}. Проводились замеры реального времени работы программы, для этого использовалась функция gettimeofday~\cite{time-city}. Замеры проводились на виртуальной машине Linux с 8~логическими ядрами, и замерялось время для дополнительных рабочих потоков в размере от 0 (вычисление в основном потоке) до 32, по степеням числа 2, то есть было рассмотрено 0, 1, 2, 4, 8, 16, 32 дополнительных рабочих потока.

Был получен график зависимости времени отрисовки сцены от числа полигонов на сцене для варьируемого числа рабочих потоков~\ref{img:graph}.

\imgScale{0.2}{graph}{Зависимость времени отрисовки сцены от числа полигонов на сцене для варьируемого числа рабочих потоков}

Из проведённых замеров были сделаны следующие выводы:
\begin{itemize}
	\item{при увеличении количества потоков время обработки сцены уменьшается, если на сцене 200 и более полигонов;}
	\item{при небольшом заполнении сцены (количество полигонов = 100) алгоритм использующий последовательную обработку данных работает быстрее, так как не тратит время на создание потоков;}
\end{itemize}


\clearpage
