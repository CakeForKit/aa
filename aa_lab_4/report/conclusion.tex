\ssr{ЗАКЛЮЧЕНИЕ}

В ходе лабораторной работы была выполнена поставленная цель, которая заключалась в исследовании зависимости времени отрисовки сцены от числа полигонов сцены для варьируемого числа рабочих потоков.


Были выполнены следующие задачи:
\begin{enumerate}
	\item реализовать алгоритма трассировки лучей для визуализации трёхмерной сцены, состоящей из моделей, заданных треугольными полигонами;
	\item исследовать зависимость времени отрисовки сцены разработанным ПО от числа полигонов сцены для варьируемого числа рабочих потоков. Изменять количество дополнительных потоков от 0 (вычисление в основном потоке), до $4\cdot k$, где $k$ --- количество логических ядер используемой ЭВМ, по степеням числа 2;
\end{enumerate}


Основываясь на проведённом исследовании были сделаны следующие выводы.
\begin{itemize}
	\item{при увеличении количества потоков время обработки сцены уменьшается, если на сцене 200 и более полигонов;}
	\item{при небольшом заполнении сцены (количество полигонов = 100) алгоритм использующий последовательную обработку данных работает быстрее, так как не тратит время на создание потоков;}
\end{itemize}
