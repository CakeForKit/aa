\aasection{Входные и выходные данные}{sec:input-output}
Входными данными программы является информация о сцене (об объектах которые на ней находятся). Модели на сцене задаются множеством треугольных полигонов. Выходными данными является растровое изображение.

\aasection{Преобразование входных данных в выходные}{sec:algorithm}
Программа определяет цвет каждого пикселя растрового изображения методом трассировки лучей.

\aasection{Примеры работы программы}{sec:demo}
На рисунке~\ref{img:img_test}-\ref{img:img_time} представлены примеры работы программы.

\imgScale{0.3}{img_test}{Пример работы программы}
\imgScale{0.3}{img_time}{Пример работы программы}
%\clearpage

\aasection{Тестирование}{sec:tests}

Для тестирования были выделены следующие классы тестирования:
\begin{enumerate}
	\item {луч не пересекается ни с одним объектом};
	\item {луч пересекается с объектом один раз};
	\item {луч и пересекается с объектом и отразившись пересекается с другим объектом};
	\item {луч пересекается с объектом в тени другого объекта};
\end{enumerate}

В таблице~\ref{tbl:tests} представлены функциональные тесты алгоритма трассировки лучей. Все тесты пройдены успешно.


\begin{longtable}{|p{.1\textwidth - 2\tabcolsep}|p{.3\textwidth - 2\tabcolsep}|p{.3\textwidth - 2\tabcolsep}|p{.3\textwidth - 2\tabcolsep}|}
	\caption{Функциональные тесты}\label{tbl:tests} \\\hline
	№ & Координаты пикселя  & Цвет пикселя (RGB) & Ожидаемый цвет пикселя (RGB)                                          \\\hline
	\endfirsthead
	\caption{Функциональные тесты (продолжение)} \\\hline
	№ & Координаты пикселя Входные данные & Цвет пикселя (RGB) & Ожидаемый цвет пикселя (RGB)                                                 \\\hline
	\endhead
	\endfoot
	1 & (190, 443) & R=0, G=0, B=0 & R=0, G=0, B=0 \\ \hline
	2 & (95, 238) & R=5, G=10, B=11 & R=5, G=10, B=11 \\ \hline
	3 & (131, 351) & R=25, G=59, B=66 &  R=25, G=59, B=66 \\ \hline
	4 & (46, 511) & R=33, G=77, B=86 & R=33, G=77, B=86 \\ \hline
\end{longtable}


\clearpage
