\chapter{Аналитическая часть}

В данной работе будут рассмотрены алгоритмы нахождения расстояния Левенштейна и Дамерау--Левенштейна.

\section{Расстояние Левенштейна}

Расстояние Левенштейна — мера различия двух последовательностей символов (строк) относительно минимального количества операций вставки, удаления и замены, необходимых для перевода одной строки в другую \cite{levenshtein}.

Пусть $S_{1}$ и $S_{2}$ - две строки, длиной \textit{M} и \textit{N} соответственно, над некоторым алфавитом, тогда расстояние Левенштейна d($s_{1}$, $s_{2}$) можно подсчитать по следующей рекуррентной формуле d($s_{1}$, $s_{2}$) = D(M, N):

\begin{equation}
	\label{eq:D}
	D(i, j) = \begin{cases}
		
		0 &\text{i = 0, j = 0}\\
		i &\text{j = 0, i > 0}\\
		j &\text{i = 0, j > 0}\\
		\min \lbrace \\
		\qquad D(i, j-1) + 1\\
		\qquad D(i-1, j) + 1 &\text{i > 0, j > 0}\\
		\qquad D(i-1, j-1) + m(s_{1}[i], s_{1}[j]) &\text(\ref{eq:m})\\
		\rbrace
	\end{cases}
\end{equation}


Функция m(a, b) определена как:
\begin{equation}
	\label{eq:m}
	m(a, b) = \begin{cases}
		0 &\text{если a = b,}\\
		1 &\text{иначе}
	\end{cases}.
\end{equation}

\section{Рекурсивный алгоритм нахождения расстояния Левенштейна}
Рекурсивный алгоритм нахождения расстояния Левенштейна реализует
формулу 1.1. Основная идея состоит в том, чтобы последовательно сравнивать символы двух строк, начиная с конца, и рекурсивно находить минимальное количество операций для преобразования предыдущих частей строк.
Рекурсивный алгоритм неэффективен, так как многие вычисления производятся повторно.

\section{Рекурсивный алгоритм нахождения расстояния Левенштейна с использованием кеша}
Рекурсивный алгоритм нахождения расстояния Левенштейна с использованием кеша является оптимизацией предыдущего алгорита. Основная идея состоит в том, чтобы избежать повторных вычислений, сохраняя уже вычисленные промежуточные результаты и используя их вместо перерасчета.

\section{Матричный алгоритм нахождения расстояния Левенштейна}
Матричный алгоритм нахождения расстояния Левенштейна работает путём построения матрицы, в которую записывается минимальное расстояние Левенштейна для подстрок. Матрица имеет размеры (N+1) * (M+1), первая строка заполняется числами от 0 до M, а первый столбец – числами от 0 до N. Ячейки матрицы заполняются формуле:
\begin{equation}
	\label{eq:mat}
	matrix[i][j] = min \begin{cases}
		matrix[i-1][j] + 1\\
		matrix[i][j-1] + 1\\
		matrix[i-1][j-1] + m(s_{1}[i], s_{2}[j])&\text(\ref{eq:m2})\\
	\end{cases}.
\end{equation}

Функция \ref{eq:m2} определена как:
\begin{equation}
	\label{eq:m2}
	m(s_{1}[i], s_{2}[j]) = \begin{cases}
		0, &\text{если $s_{1}[i - 1] = s_{2}[j - 1]$,}\\
		1, &\text{иначе}
	\end{cases}.
\end{equation}

Результат вычисления расстояния Левенштейна будет в ячейке матрицы с индексами $i = N$ и $j = M$.

\section{Алгоритм нахождения расстояния Дамерау--Левенштейна}
Расстояние Дамерау--Левенштейна является модификацией расстояния Левенштейна: к операциям вставки, удаления и замены символов, определённых в расстоянии Левенштейна добавлена операция транспозиции (перестановки) символов.

Расстояние Дамерау--Левенштейна может быть вычислено по рекуррентной формуле:

\begin{equation}
	\label{eq:DL}
	D(i, j) = 
	\begin{cases}
		0, &\text{i = 0, j = 0,}\\
		i, &\text{j = 0, i > 0,}\\
		j, &\text{i = 0, j > 0,}\\
		min \begin{cases}
			D(i, j - 1) + 1,\\
			D(i - 1, j) + 1,\\
			D(i - 1, j - 1) + m(S_{1}[i], S_{2}[j])&\text(\ref{eq:m2})\\, \\
			D(i - 2, j - 2) + 1, \\
		\end{cases}
		& \begin{aligned}
			& \text{если i > 1, j > 1}, \\
			& S_{1}[i] = S_{2}[j - 1], \\
			& S_{1}[i - 1] = S_{2}[j], \\
		\end{aligned}\\
		min \begin{cases}
			D(i, j - 1) + 1,\\
			D(i - 1, j) + 1, \\
			D(i - 1, j - 1) + m(S_{1}[i], S_{2}[j])&\text(\ref{eq:m2})\\, \\
		\end{cases}
		& \text{иначе.}
	\end{cases}
\end{equation}


\section*{Вывод}
В данном разделе были теоретически разобраны рекуррентная формула расстояния Левенштейна и различные ее модификации и формула расстояния Дамерау--Левенштейна.


\clearpage
