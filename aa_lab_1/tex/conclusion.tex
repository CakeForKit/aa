\ssr{ЗАКЛЮЧЕНИЕ}


В ходе лабораторной работы была выполнена поставленнная цель, которая заключалась в выполнении сравнительного анализа алгоритмов нахождения расстояний Левенштейна и Дамерау–-Левенштейна.
Были выполнены следующие задачи:
\begin{enumerate}
	\item были рассмотрены алгоритмы Левенштейна и \\ Дамерау--Левенштейна нахождения расстояния между строками;
	\item были реализованы следующие алгоритмы;
	\begin{itemize}
		\item Рекурсивный алгоритм нахождения расстояния Левенштейна;
		\item Рекурсивный алгоритм нахождения расстояния Левенштейна с использованием кеша;
		\item Матричный алгоритм нахождения расстояния Левенштейна;
		\item Алгоритм нахождения расстояния Дамерау--Левенштейна;
	\end{itemize}
	\item был проведен сравнительный анализ алгоритмов по времени;
\end{enumerate}


Основываясь на проведенном исследовании можно сделать следующие выводы:
\begin{itemize}
	\item{Матричная реализация алгоритма Левенштейна демонстрирует наилучшие результаты по времени работы на всех тестовых данных.}
	\item{Рекурсивная реализация алгоритма Левенштейна без мемоизации значительно проигрывает по времени всем другим подходам.}
\end{itemize}
