\chapter{Технологическая часть}
В данном разделе будут приведены средства реализации, листинг кода и функциональные тесты.

\section{Средства реализации}
В данной работе для реализации был выбран язык программирования $Python$\cite{python-city}, так как он удовлетворяет требованиям лабраторной работы: может замерить процессорное время (с помошью функции $process\_time\_ns(...)$ из библиотеки $time$\cite{python-time})

\section{Реализация алгоритмов}
В листингах \ref{lst:lev_rec}-\ref{lst:damerau_lev_matrix} представлены реализации алгоритмов нахождения расстояния Левенштейна и Дамерау--Левенштейна.

\begin{center}
	\captionsetup{justification=raggedright,singlelinecheck=off}
	\renewcommand{\lstlistingname}{Листинг}
	\begin{lstlisting}[label=lst:lev_rec,caption=Рекурсивный алгоритм нахождения расстояния Левенштейна]
		def lev_recurs_(s1, s2, i, j):
			if i == 0:
				return j
			if j == 0:
				return i
			
			flag = 0 if s1[i - 1] == s2[j - 1] else 1
			
			variants = [
				lev_recurs_(s1, s2, i, j - 1) + 1,
				lev_recurs_(s1, s2, i - 1, j) + 1,
				lev_recurs_(s1, s2, i - 1, j - 1) + flag
			]
			return min(variants)
		
		
		def lev_recurs(s1, s2):
			return lev_recurs_(s1, s2, len(s1), len(s2))
	\end{lstlisting}
\end{center}

\begin{center}
	\captionsetup{justification=raggedright,singlelinecheck=off}
	\renewcommand{\lstlistingname}{Листинг}
	\begin{lstlisting}[label=lst:lev_cache, caption=Рекурсивный алгоритм нахождения расстояния Левенштейна с использованием кеша]
	def lev_cache_matrix(s1, s2, i, j, matrix):
		if matrix[i][j] != -1:
			return matrix[i][j]
		
		if i == 0:
			matrix[i][j] = j
			return j
		if j == 0:
			matrix[i][j] = i
			return i
		
		flag = 0 if s1[i - 1] == s2[j - 1] else 1
		variants = [
			lev_cache_matrix(s1, s2, i, j - 1, matrix) + 1,
			lev_cache_matrix(s1, s2, i - 1, j, matrix) + 1,
			lev_cache_matrix(s1, s2, i - 1, j - 1, matrix) + flag,
		]
		matrix[i][j] = min(variants)
		
		return matrix[i][j]
	
	
	def lev_cache(s1, s2, printing=False):
		i, j = len(s1), len(s2)
		matrix = []
		for i in range(len(s1) + 1):
			matrix.append([-1] * (len(s2) + 1))
		
		res = lev_cache_matrix(s1, s2, i, j, matrix)
		if printing:
			for i in range(len(s1) + 1):
				print(matrix[i])
		return res
	\end{lstlisting}
\end{center}

\begin{center}
	\captionsetup{justification=raggedright,singlelinecheck=off}
	\renewcommand{\lstlistingname}{Листинг}
	\begin{lstlisting}[label=lst:lev_mat, caption=Матричный алгоритм нахождения расстояния Левенштейна]
	def lev_mat(s1, s2, printing=False):
		mat = [[i for i in range(len(s2) + 1)]]
		if printing: print(mat[0])
		for i in range(len(s1)):
			mat.append([i + 1])
			i += 1
			for j in range(1, len(s2) + 1):
				flag = 0 if s1[i - 1] == s2[j - 1] else 1
				new = min(
					mat[i][j - 1] + 1,
					mat[i - 1][j] + 1,
					mat[i - 1][j - 1] + flag
				)
		
		mat[i].append(new)
		
		if printing: print(mat[i])
			return mat[-1][-1]
	\end{lstlisting}
\end{center}

\begin{center}
	\captionsetup{justification=raggedright,singlelinecheck=off}
	\renewcommand{\lstlistingname}{Листинг}
	\begin{lstlisting}[label=lst:damerau_lev_matrix, caption=Алгоритм нахождения расстояния Дамерау--Левенштейна]
		def damerau_lev_matrix(s1, s2, i, j, matrix):
		if matrix[i][j] != -1:
			return matrix[i][j]
		
		if i == 0:
			matrix[i][j] = j
			return j
		if j == 0:
			matrix[i][j] = i
			return i
		
		flag = 0 if s1[i - 1] == s2[j - 1] else 1
		variants = [
			damerau_lev_matrix(s1, s2, i, j - 1, matrix) + 1,
			damerau_lev_matrix(s1, s2, i - 1, j, matrix) + 1,
			damerau_lev_matrix(s1, s2, i - 1, j - 1, matrix) + flag,
		]
		if i > 1 and j > 1 and s1[i - 1] == s2[j - 2] and s1[i - 2] == s2[j - 1]:
			variants.append(
				damerau_lev_matrix(s1, s2, i - 2, j - 2, matrix) + 1
			)
		matrix[i][j] = min(variants)
		
		return matrix[i][j]
		
		
		def damerau_lev(s1, s2, printing=False):
			i, j = len(s1), len(s2)
			matrix = []
			for i in range(len(s1) + 1):
				matrix.append([-1] * (len(s2) + 1))
			
			res = damerau_lev_matrix(s1, s2, i, j, matrix)
			if printing:
				for i in range(len(s1) + 1):
					print(matrix[i])
			return res
	\end{lstlisting}
\end{center}


\section{Классы эквивалентности тестирования}

Для тестирования были выделены следующие классы тестирования:
\begin{enumerate}
	\item {Пустые строки}
	\item {Одна строка пустая, другая нет}
	\item {Строки равной длины}
	\item {Анаграммы} 
	\item {Расстояния, которые вычислены алгоритмами Левенштейна и ДамерауЛевенштейна, равны}
	\item {Расстояния, которые вычислены алгоритмами Левенштейна и ДамерауЛевенштейна, дают разные результаты}
\end{enumerate}

\section{Функциональные тесты}
В таблице \ref{tbl:functional_test} приведены функциональные тесты для алгоритмов нахождения расстояния Левенштейна и Дамерау--Левенштейна. 

Все тесты пройдены успешно.

\FloatBarrier
\begin{table}[h]
	\begin{center}
		\begin{threeparttable}
			\captionsetup{justification=raggedright,singlelinecheck=off}
			\caption{\label{tbl:functional_test} Функциональные тесты}
			\begin{tabular}{|c|c|c|c|c|}
				\hline
				& & & \multicolumn{2}{c|}{Результат} \\
				\hline
				№ & Строка 1 & Строка 2 & Левенштейн & Дамерау--Л. \\
				\hline
				1&"Пустая строка"&"Пустая строка"&0&0 \\
				\hline
				2&"Пустая строка"&Слово&5&5 \\
				\hline
				3&Мука&"Пустая строка"&4&4 \\
				\hline
				4&мука&река&2&2 \\
				\hline
				5&абвг&гбав&3&3 \\
				\hline
				6&123&123456&3&3 \\
				\hline
				7&1234&2134&2&1 \\
				\hline
			\end{tabular}
		\end{threeparttable}
	\end{center}
\end{table}
\FloatBarrier


\section*{Вывод}

Были представлены листинги всех описанных ренее алгоритмов нахождения расстояния Левенштейна и Дамерау--Левенштейна и их тесты.

\clearpage
