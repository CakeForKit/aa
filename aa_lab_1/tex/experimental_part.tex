\chapter{Исследовательская часть}

Цель исследования - определение зависимости времени работы алгоритмов нахождения расстояния Левенштейна и Дамерау-Левенштейна от размера поданых на вход строк.


\section{Технические характеристики}
Технические характеристики устройства, на котором выполнялось тестирование:
\begin{itemize}
	\item {Операционная система – Майкрософт Windows 11 Домашняя для одного языка; Версия - 10.0.22631; Сборка - 22631;}
	\item {Установленная оперативная память (RAM) - 16,0 ГБ;}
	\item {Процессор - AMD Ryzen 7 5800H with Radeon Graphics, 3201 МГц, ядер: 8, логических процессоров: 16;}
\end{itemize}

При тестировании ноутбук был включен в сеть электропитания, Во время тестирования на ноутбуке были запущены только встроенное приложение окружения PyCharm и система тестирования.

\section{Время выполнения алгоритмов}
Как было сказано выше для замера процессорного времени использовалась функция process\_time\_ns(...) из библиотеки time на Python.

Для графика \ref{img:graph4} замеры проводились для длины слов от 0 до 12. А для графика \ref{img:graph3} для длины слов от 0 до 500, так как время работы рекурсивного алгоритма возрастает, намного быстрее, чем остальных.

Замеры времени проводились по принципу: для одних входных данные проводилось 10 замеров и если относительная стандартная ошибка среднего (rse) была >= 5\%, то для этих данных замеры продолжались, в таблицу заносилось среднее значение.

\imgScale{0.8}{graph4}{Сравнение по времени алгоритмов Левенштейна и Дамерау–Левенштейна}

\imgScale{0.8}{graph3}{Сравнение по времени алгоритмов Левенштейна с использованием кеша, матричной реализации и Дамерау–Левенштейна}

\clearpage

\section{Вывод}

Из проведённых замеров можно сделать следующие выводы:
\begin{itemize}
	\item{Матричная реализация алгоритма Левенштейна демонстрирует наилучшие результаты по времени работы на всех тестовых данных.}
	\item{Рекурсивная реализация алгоритма Левенштейна без мемоизации значительно проигрывает по времени всем другим подходам.}
\end{itemize}

\clearpage
