\chapter{Конструкторская часть}
В этом разделе будут представленs схемы алгоритмов вычисления расстояния Левенштейна и Дамерау--Левенштейн.
\section{Описание алгоритмов}
На рисунках \ref{img:lev_recurs}-\ref{img:damerau_lev} представлены схемы алгоритмов вычисления расстояния Левенштейна и Дамерау--Левенштейна.

\imgScale{0.6}{lev_recurs}{Схема рекурсивного алгоритма нахождения расстояния Левенштейна}
\clearpage

\imgScale{0.6}{lev_cache_matrix}{Схема рекурсивного алгоритма нахождения расстояния Левенштейна с использованием кеша (матрицы)}
\clearpage

\imgScale{0.6}{lev_mat_1}{Схема матричного алгоритма нахождения расстояния Левенштейна}
\clearpage

\imgScale{0.6}{lev_mat_2}{Схема матричного алгоритма нахождения расстояния Левенштейна}
\clearpage

\imgScale{0.6}{damerau_lev}{Схема алгоритма нахождения расстояния Дамерау--Левенштейна}
\clearpage

%\section{Описание используемых типов данных}
%При реализации алгоритмов будут использованы следующие структуры данных:

%\begin{itemize}
%	\item строка типа \textit{str};
%	\item длина строки - целое число типа \textit{int};
%	\item матрица — двумерный массив значений типа \textit{int}.
%\end{itemize}


\section*{Вывод}

В данном разделе были представлены схемы алгоритмов нахождения расстояний Левенштейна и Дамерау--Левенштейна.

\clearpage
